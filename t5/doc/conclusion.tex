\section{Conclusion}
\label{sec:conclusion}

In this laboratory assignment the objective of simulating a Band Pass Filter (BPF) using OPAMP has been achieved. The theoretical analysis was performed with the help of the Octave math tool and the circuit simulation using the Ngspice tool. For both analysis, we determined the gain ,the input and the output impedances of the circuit using a central frequency, as well as, the gain and phase frequency response for an array of frequency values that range from 10 Hz to 100MHz, the results were also plotted for this values.At the end, we calculate the merit of our work.

This way, in theoretical analysis, we explained and compute the results obtained using an ideal OPAMP with the help of the math tool previously mentioned and the formulas that we learned in the theoretical lectures. And, in simulation analysis, we compare both the simulation and the theoretical results, explaining why the results are similar or differ.

 As previously mentioned, the simulation results had slight differences from the theoretical ones. However, we designed an acceptable Band Pass Filter - furthermore, we believe that the objective of this assignment was achieved with great success.

\section{Theoretical Analysis}
\label{sec:analysis}

In this section, the circuit shown in Figure~\ref{fig:circuit} is analysed
theoretically. We will begin by analyzing the Envelope Detector circuit and, after that, the Voltage regulator circuit, in order to predict their outputs.

The theoretical values will be obtained by applying Kirchhoff laws and the diode equations.
Considering the circuiy of Figure~\ref{fig:circuit}, is composed by a Voltage source, a transformer, an envelope detector and a voltage regulator.

The amplitude and the frequency of the voltage source are equal to 230V and 50Hz, respectively. However, the transformer will convert this into a lower voltage, supllying the rest of the circuit with a voltage, Vs(t):
\begin{equation}
  V_s(t)= A \cdot cos(wt)
  \label{eq:Vs(t)}
\end{equation}
, where
\begin{equation}
  w= 2 \cdot \pi \cdot f
  \label{eq:Vs(t)}
\end{equation}

\subsection{Envelope Detector}
The envelope detector consists of a rectifier, composed by four diodes, a resistor, R1, and a capacitor, C.

The rectifier used on this AC/DC converter was a full-wave rectifier, more specifically, a bridge rectifier. Theoretically, this rectifier produce an output of $V_(rectifier)=|V_s(t)|$.

The resistor and the capacitor are used to smooth the wave.



\subsection{Voltage Detector}

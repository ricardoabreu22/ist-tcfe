\section{Simulation Analysis}
\label{sec:simulation}

\subsection{Operating Point Analysis}

	Table~\ref{tab:op} shows the simulated operating point results for the circuit
under analysis. Compared to the theoretical analysis results, we notice that the
simulation results are accurate, except for the last decimal places, 
as a consequence of the cientific notation and the number of significative algharisms 
used by each program to present the results. Despite that, we realise that the 
values with more significant algharisms (used in NGSpice) match correctly the 
rounded values (used in Octave).


\begin{table}[h]
  \centering
  \begin{tabular}{|l|r|}
    \hline    
    {\bf Name} & {\bf Value [A or V]} \\ \hline
    @c[i] & 0.000000e+00\\ \hline
@gb[i] & -2.24935e-04\\ \hline
@r1[i] & 2.149178e-04\\ \hline
@r2[i] & -2.24935e-04\\ \hline
@r3[i] & -1.00169e-05\\ \hline
@r4[i] & 1.233378e-03\\ \hline
@r5[i] & -2.24935e-04\\ \hline
@r6[i] & 1.018460e-03\\ \hline
@r7[i] & 1.018460e-03\\ \hline
v1 & 5.248421e+00\\ \hline
v2 & 5.033187e+00\\ \hline
v3 & 4.581562e+00\\ \hline
v4 & -2.05008e+00\\ \hline
v5 & 5.064367e+00\\ \hline
v6 & 5.745178e+00\\ \hline
v7 & -2.05008e+00\\ \hline
v8 & -3.09813e+00\\ \hline

  \end{tabular}
  \caption{Operating point. A variable preceded by @ is of type {\em current}
    and expressed in Ampere; other variables are of type {\it voltage} and expressed in
    Volt.}
  \label{tab:op}
\end{table}

In the table, we can see an nineth node (node 8), which has the same voltage value as node 6. This happens because, in NGSPice, when we want to simulate circuits with current-dependent sources, we must add a 0V voltage source in series to a component to sense the current flowing through it. Therefore, an aditional node appears in the simulated circuit.

Note that we can not perform aditional simulation analysis, namely transient and frequency ones, with phase and magnitude responses and input impedance, because the circuit does not have any electrical component which output is a function of time.
